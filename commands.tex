\DeclareMathOperator{\lengthofmodule}{\mathrm{length}} % Krull dimension
\DeclareMathOperator{\transcedentaldegree}{\mathrm{tr.deg}} % transcedental degree
\DeclareMathOperator{\heightofprime}{\mathrm{ht}} % height of prime ideal
\newcommand[1]{\height}{\heightofprime\mleft(#1\mright)}
\DeclareMathOperator{\Kdim}{\mathrm{K-dim}} % Krull dimension
\DeclareMathOperator{\topdim}{\mathrm{dim}} % topological dimension
\DeclareMathOperator{\WeilDivisor}{\mathrm{Div}} % Weil divisor
\newcommand[1]{\WDiv}{\WeilDivisor\mleft(#1\mright)}
\DeclareMathOperator{\CartierDivisor}{\mathrm{CDiv}} % Cartier divisor
\newcommand[1]{\CDiv}{\CartierDivisor\mleft(#1\mright)}
