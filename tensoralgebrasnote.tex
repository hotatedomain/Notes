\DeclareMathOperator{\Der}{Der}
\newcommand{compo}{\circ}
\newcommand{\diffmod}{\Omega}
\section{微分加群}
\begin{defi}[導分]
\(A\)を環,\(B\)を\(A\)-代数,\(M\)を\(B\)-加群とする.\(A\)-加群の射\(D\in\Hom_A(B,M)\)であって
\[D(bb')=bD(b')+b'D(b)\]
を満たすものを,\(B\)から\(M\)への\(A\)上の導分(\(A\)-derivation)という.
\end{defi}
\(B\)から\(M\)への\(A\)上の導分全体の集合を\(\Der_A(B,M)\)と書く.これには\(B\)-加群の構造が自然に入る.\(A=\Z\)のとき,単に\(\Der(B,M)\)と書く.\(M=B\)のとき,単に\(\Der_A(B)\)と書く.
\begim{lem}
任意の\(D\in\Der_A(B,M)\)に対し\(D(1)=0\)であるから,特に\(D(A)=0\)である.従って\(A\)-代数の構造射が全射なら\(\Der_A(B,M)=0\)である.
\end{lem}
環\(C=B\otimes_AB\)には\(b(b_1\otimes b_2)=(bb_1)\otimes b_2\)により\(B\)-代数の構造が入ることに注意する.\(A\)-代数の射\(\mu\colon C\to B;\ b\otimes b' \mapsto bb'\)に対し,\(I_{B/A}=\Ker(\mu)\)とおく.\(I_{B/A}\)は\(a\otimes b-b\otimes1\)という形の元で生成される.\(B\otimes_AB\\)は\(B\)-代数とみなすことができ,この構造射によって\(I_{B/A}\)は\(B\)-加群とみなせる.
\begin{defi}[微分加群]
剰余\(B\)-加群\(I_{B/A}/I_{B/A}^2\)を\(\Omega_{B/A}^1\)と書き,\(B\)の\(A\)上の微分加群(differential module)という.また,\(d=d_{B/A}\in\Hom(B,\Omega_{B/A}^1)\)を
\[db=1\otimes b-b\otimes1\mod I_{B/A}^2\]
で定める.
\end{defi}
\(d_{B/A}\)は実は\(A\)-加群の射になる.実際,\(a\in A\), \(b\in B\)に対して
\[d(ab)=1\otimes(ab)-(ab)\otimes1=(1\otimes a)(1\otimes b-b\otimes1)=(1\otimes a)db\]
である.\(\Omega_{B/A}^1\)は\(B\)-加群としては\(\Sets{db | b\in B}\)によって生成される.

微分加群は導分の表現対象として特徴づけられる.\(B\)-加群の圏\(\Mod{B}\)に対し,対応
\[\Der_A(B,-)\colon M\mapsto\Der_A(B,M)\]
は\(\Mod{B}\)から\(\Mod{B}\)への共変関手になっている.この関手が微分加群で表現されることを示そう.
\begin{theo}
任意の\(B\)加群\(M\)と\(A\)上の導分\(D\in \Der_A(B,M)\)に対して,\(D=f\compo d_{B/A}\)を満たす\(B\)加群の射\(f\colon \Omega_{B/A}^1\to M\)が一意的に存在する.特に,関手的な\(B\)-加群の同形
\[\Hom_B(\Omega_{B/A}^1,M)\cong\Der_A(B,M\)\]
が存在する.
\end{theo}
\begin{proof}

\end{proof}

いくつか微分加群を計算してみる.典型的には多項式環\(B=A[x_1,\dots,x_n]\)の場合が微分形式の代数的類似になっている.
\begin{exam}
\(B=A[x_1,\dots,x_n]\)のとき,\(f\in B\)に対して
\[df=\sum_{i=1}^n\frac{\partial f}{\partial x_i}dx_i\]
である.
\end{exam}

\section{テンソル代数}
この節では\(A\)は可換環とする.
\begin{defi}
\(A\)-加群\(M\)と非負整数\(k\)に対し,\(T_A^k(M)\coloneqq M^{\otimes k}=M\otimes M\otimes\dots\otimes M\)とおき,\(M\)の\(k\)次テンソル冪という.
ただし,\(T_A^0(M)=A\)とする.

次に,これらテンソル冪の直和\(T_A(M)\coloneqq \bigoplus_{k\leq0}T^k(M)=A\oplus M\oplus M^{\otimes2}\oplus\dots\)を\(M\)のテンソル代数という.
\end{defi}
係数環\(A\)が明らかなときは\(T^p(M)\)や\(T(M)\)と書き表す.非負整数\(p\), \(q\)に対し,テンソル積の誘導する同形\(T^p(M)\otimes T^q(M)\to T^{p+q}(M)\)によって積が定まる.すなわち,
\[(x_1\otimes\dots\otimes x_p)\cdot(y_1\otimes\dots\otimes y_q)=x_1\otimes\dots\otimes x_p\otimes y_1\otimes\dots\otimes y_q\]
を線形に拡張して定める.\(T_A(M)\)には次数付き\(A\)-代数の構造が入る.

\begin{defi}[外積代数]
\(M\)の外積代数\(\exterialg(M)\)を,対称代数\T(M)\)の\(x\otimes x\in T^2(M)\)が生成するイデアルによる商として定める.

\(x_1\otimes\dots\otimes x_k\in T^k(M)\)の像を\(x_1\wedge\dots\wedge x_k\in \wedge^kM\)と書く.
\end{defi}
